\documentclass[11pt]{article}
% packages
\usepackage{graphicx}
\graphicspath{ {img/} }

\usepackage{fontspec}
\usepackage{multicol}
\usepackage{vwcol}
\usepackage{titlesec}
\usepackage{outlines}
\usepackage{tabularx}
\usepackage{booktabs}
\usepackage{multirow}
\usepackage{enumitem}

\usepackage[paper=a4paper,left=10mm,right=10mm,top=20mm,bottom=0mm]{geometry}

\pagestyle{empty}

% fonts

\newfontfamily{\robotolight}{[Roboto-Light.ttf]}
\newfontfamily{\robotocondlight}{[RobotoCondensed-Light.ttf]}
\newfontfamily{\robotocond}{[RobotoCondensed-Regular.ttf]}
\newfontfamily{\worksanslight}{[WorkSans-Light.ttf]}
\newfontfamily{\worksans}{[WorkSans-Regular.ttf]}

% \renewcommand{\familydefault}{\robotocondlight}

% section headers
\usepackage{sectsty}
\sectionfont{\fontsize{13}{15}\worksans}
\titlespacing{\section}{0pt}{*0}{*0}
\newcommand{\sechead}[1]{
    \noindent{\fontsize{13}{15}\worksans #1} \hfill \par
}

% format the header
\newcommand{\myHeader}[2]{
    \noindent
    {\fontsize{45}{52}\worksanslight #1} \hfill \par
    \vspace{5mm}
    \hspace{15mm}{\fontsize{22}{24}\robotolight #2} \hfill \par
}

% item for contacts list
\newcommand{\contactItem}[2]{
    {\fontsize{8}{10}\robotocondlight #1} \includegraphics[width=3mm,scale=0.5]{#2} \par
}

% course entry in relevant coursework
\newcommand{\courseentry}[2]{
    {\robotocond \tabitem #1:} & #2
}

% spacefiller for empty course (pass over multiple liners)
\newcommand{\emptycourse}[0]{
    & 
}

% tabularx column that has automatic sizing but left justification
\newcolumntype{L}[0]{>{\RaggedRight} X}

\newcommand{\projectheader}[3]{
    \noindent{\fontsize{11}{13}\robotocond #1}{\fontsize{10}{12}\robotocondlight \tabitem #2 \hfill #3}
}

\newcommand{\expheader}[5]{
    \noindent #1 \tabitem #2 \tabitem #3 \hfill #4 - #5
}

% horizontal separator
\newcommand{\hsep}[1][\medskipamount]{\par
    \begin{center}
    \vspace*{\dimexpr-\parskip-\baselineskip+#1}
    \noindent\rule{190mm}{0.4pt}\par
    \vspace*{\dimexpr-\parskip-.5\baselineskip+#1}
    \end{center}
}

% large horizontal separator
\newcommand{\bighsep}[1][\medskipamount]{\par
    \begin{center}
    \vspace*{\dimexpr-\parskip-\baselineskip+#1}
    \noindent\rule{190mm}{3pt}\par
    \vspace*{\dimexpr-\parskip-.5\baselineskip+#1}
    \end{center}
}

% make a bullet
\newcommand{\tabitem}{~~\llap{\textbullet}~~}


\begin{document}
\myHeader{Thomas Varano}
% contact
\vspace{-3mm}
\begin{center}
    \contactItem{thvarano@gmail.com}{mailto:thvarano@gmail.com} |
    \contactItem{tomvarano.com}{https://tomvarano.com} |
    \contactItem{(201) 887-3953}{tel:2018873953} |
    \contactItem{linkedin.com/in/thomas-varano/}{https://linkedin.com/in/thomas-varano} |
    \contactItem{github.com/tvarano}{https://github.com/tvarano} 
\end{center}
\bighsep[1pt]

% top row
\begin{tabular}[t]{l | l}
    \hspace{-8mm}
    \noindent\parbox[t][][t]{0.3\textwidth}{
        % education
        \setlength\topsep{1pt}
        \sechead{Education}
        {\fontsize{9}{12}\robotocondlight
        \begin{itemize}[noitemsep, topsep=0pt, label={}, leftmargin=*]
            \item University of Maryland – College Park
            \item BS Computer Science
            \item Minor in Mathematics
            \item Graduation: Cum Laude May 2022
            \item GPA: 3.91
        \end{itemize}
        }
    }
    &
    \parbox[t][][t]{0.4\textwidth}{
        % Relevant Coursework
        \sechead{Relevant Coursework}
        {\fontsize{9}{12}\robotocondlight
        \noindent\begin{tabularx}{\textwidth}[t]{l l l}
            % \courseentry{Object Oriented Programming II} & \courseentry{Linear Algebra} & \courseentry{Intro to Computer Systems} \\
            \courseentry{Discrete Structures} & \courseentry{Calculus III} & \courseentry{Organization of Programming Languages} \\
            \courseentry{Applications of Linear Algebra} & \courseentry{Algorithms} & \courseentry{Advanced Data Structures} \\
            \courseentry{Applied Probability and Statistics} & \courseentry{Intro to Data Science} & \courseentry{Analysis of Computer Algorithms} \\
            \courseentry{Intro to Machine Learning} & \courseentry{Computer Networks} & \courseentry{Advanced Calculus I} \\
            \courseentry{Programming Languages and Paradigms} & \courseentry{Computer Vision} & \courseentry{Number Theory}
        \end{tabularx}
        }
    }
    
\end{tabular}
\hsep[1pt]
\vspace{-2mm}

% skills
\setlength\topsep{0pt}
\sechead{Skills}
{\fontsize{10}{12}\robotocondlight
    \noindent{\robotocond Programming Languages (Proficient:)} Python, Swift, Java \quad {\robotocond (Experience:)} C, Rust, OCaml, JavaScript, BASH, SQLite, Ruby, Assembly (MIPS)\par
    \noindent{\robotocond Technologies:}
    Git, XCTest, \LaTeX, HTML / CSS, Flask, Google Cloud, JUnit, Jupyter, Windows, MacOS, Linux \par
}

\hsep

% Experience
\sechead{Experience}
\vspace{2mm}
{\fontsize{10}{12}\robotocondlight
% Apple
    \expheader{Software Engineering Intern}{Apple}{May 2021}{August 2021}
    \begin{itemize}[noitemsep,nolistsep]
        \item Validated syncing capabilities of the Shortcuts app on iOS and macOS. 
        \item Quickly comprehended and navigated expansive codebase, developing new tests and improving structure they are built on. 
        \item Developed MultiDevice testing approach to sync multiple devices in a test run, using Swift to run custom Test Plans.
        \item Triaged test results, report specific failures to Shortcuts developers to improve product to ship.
    \end{itemize}
    \vspace{3mm}
% TA
    \expheader{Teaching Assistant}{University of Maryland}{August 2020}{May 2022}
    \begin{itemize}[noitemsep,nolistsep]
        \item Taught student-let course CMSC 389O, The Coding Interview. Teach students techniques for behavioral and technical interviews. 
        \item Acted as office hours and grading teaching assistant for CMSC 132, Object Oriented Programming II.
        \item Held virtual office hours for students, teaching data structures in Java and helping students debug projects.
    \end{itemize}
    \vspace{3mm}

% Fidelity
    \expheader{IT Intern}{Fidelity National Financial}{July 2020}{August 2020}
    \begin{itemize}[noitemsep,nolistsep]
        \item Imaged devices, set up workstations, configure print servers. 
        \item Troubleshot various server and user issues either in person or over remote connection.
        \item Learned intricacies of numerous software programs regarding installation, troubleshooting, and repair, maintaining the effectiveness of coworkers. 
    \end{itemize}
    \vspace{3mm}

\vspace{-2mm}
% {\fontsize{8}{10}\robotocondlight References available upon request.}
% }
\hsep


% Projects
\sechead{Projects}
\vspace{2mm}
\projectheader{Downpou-rs}{School Project}{2021}
\begin{itemize}[noitemsep,nolistsep]
    \item A Bittorrent client built from scratch in {\robotocond Rust}.
    \item Comprehended and utilized protocol specifcations to manually send requests to peers and trackers.
    \item Client able to communicate with HTTP trackers and download / seed with optimistic unchoking and peer ranking.
\end{itemize}
\projectheader{MultiDevice Testing}{Intern Project}{2021}
\begin{itemize}[noitemsep,nolistsep]
    \item Intern project to improve Multi Device testing at Apple. 
    \item Created a new approach for Multi Device testing, generalized to be used on multiple platforms for multiple different Apple products. 
    \item Created a suite of tests in {\robotocond Swift} using {\robotocond XCodeUI Tests} for the Shortcuts app, programmatically verifying the app's syncing feature.
    \item Ran newly created tests in CI environment for Presubmission testing.
    \item Thoroughly documented project for ease of use across teams and software implementations.
\end{itemize}

\projectheader{Offset}{Competition Project}{2021}
\begin{itemize}[noitemsep,nolistsep]
    \item A Bitcamp hackathon project analyzing the carbon footprint of consumer deliveries and online shopping. 
    \item Parsed package tracking numbers from user's Gmail using {\robotocond Python} API. 
    \item Reverse engineered APIs to gather information about packages and their exact carbon impacts. 
    \item Used {\robotocond Python Flask} to host a website querying all endpoints and displaying information to user.
\end{itemize}
}
{\fontsize{8}{10}\robotocondlight Project links: tomvarano.com/about \hfill Github: github.com/tvarano
}
\hsep
% Activities
\sechead{Activities / Leadership}
{\fontsize{10}{12}\robotocondlight
\noindent Open Sourcery
\begin{itemize}[noitemsep,nolistsep]
    \item Computer Science club specializing in open source cooperation and contribution. Weekly meetings either involve workshops on varying 
    technologies or breakoffs where different groups within the club discuss and work on their own team projects. 
\end{itemize}
Sigma Phi Delta
\begin{itemize}[noitemsep,nolistsep]
    \item Engineering Fraternity
    \item Webmaster: Maintain website, approve pull requests changes from other chairs for their respective pages.
    \item Assistant Treasurer: Delegate spending, budget income across projects and events.
    \item Risk Manager: Minimize Covid risk and ensure all events adhere to Covid guidelines. 
\end{itemize}
Running Club, University Jazz Band, Terrapin Ski Club, UMD Spikeball Club
\end{document}
